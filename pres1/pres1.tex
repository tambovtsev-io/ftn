\documentclass{beamer}
\usepackage[T1, T2A]{fontenc}
\usepackage[utf8]{inputenc}
\usepackage[russian, english]{babel}
\usepackage{cmap}
\usepackage{subcaption}
\usepackage{float}

\usepackage{gensymb}
\usepackage{array}
\usepackage{amsmath}
\usepackage{amssymb}
\usepackage{wrapfig}

\usepackage{tabu, booktabs}
\usepackage{makecell}

\usepackage{graphicx}
\hypersetup{unicode=true}

\graphicspath{ {./img/} }

\usetheme{Madrid}
\usefonttheme[onlymath]{serif}

\title{Работа по поляризации}
\subtitle{}

%\usetheme{lucid}
\begin{document}
\frame {
    \titlepage
}


\frame {
\frametitle{Теория для первой части}
Поляризация -- зависимость направления колебаний электрического поля в
электромагнитной волне от времени. В работе мы имеем дело с линейной и
круговой поляризацией.

Линейная поляризация -- вектор $\vec{E}$ колеблется в одной плоскости.

Круговая -- разность фаз между колебаниями компонент $E_y$ и $E_x$ равна $\pi/2$
}


\frame{
    \frametitle{Sample Page 2}
    \framesubtitle{An Example of Lists}
    \begin{itemize}
        \item 1
        \item 2
        \item 3
    \end{itemize}
}


\frame{
    \frametitle{Paragraph Content}
    This is a paragraph.
}
\end{document}